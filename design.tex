\documentclass{article}
\usepackage{times}
\usepackage{xspace}
\usepackage{color}
\usepackage{listings}
\usepackage{setspace}
\usepackage{url}
\usepackage{longtable}

\newcommand{\lapps}{LAPPS\xspace}
\newcommand{\data}{\texttt{Data}\xspace}
\newcommand{\source}{\texttt{DataSource}\xspace}
\newcommand{\service}{\texttt{WebService}\xspace}
\newcommand{\discriminator}{\textit{discriminator}\xspace}
\newcommand{\payload}{\textit{payload}\xspace}
\newcommand{\vocab}[1]{\url{http://vocab.lappsgrid.org/#1}}
\newcommand{\ns}[1]{\url{http://vocab.lappsgrid.org/ns/#1}}

\newcommand{\definedterm}[1]{\textbf{\textit{#1}}\xspace}
\newcommand{\must}{\definedterm{must}}
\newcommand{\mustnot}{\definedterm{must not}}
\newcommand{\should}{\definedterm{should}}
\newcommand{\shouldnot}{\definedterm{should not}}
\newcommand{\may}{\definedterm{may}}
\newcommand{\maynot}{\definedterm{may not}}


\renewcommand{\tt}[1]{\texttt{#1}}
\newcommand{\bold}[1]{\textbf{#1}}

\lstnewenvironment{groovy}[2]
  {\singlespacing\lstset{language=java, label=#1, caption=#2, captionpos=b, tabsize=3}}
  {}

\newenvironment{listing}{
\begin{itemize}
  \setlength{\itemsep}{1pt}
  \setlength{\parskip}{0pt}
  \setlength{\parsep}{0pt}
}{\end{itemize}}

\newenvironment{enum}{
\begin{enumerate}
  \setlength{\itemsep}{1pt}
  \setlength{\parskip}{0pt}
  \setlength{\parsep}{0pt}
}{\end{enumerate}}


\begin{document}

\title{Design of the Language Application Grid}
\author{Keith Suderman\\
Department of Computer Science\\
Vassar College\\
Poughkeepsie, New York\\
\texttt{suderman@cs.vassar.edu}}
\date{\today}
\maketitle

\begin{abstract}
The Language Application (LAPPS) Grid project
is establishing a
framework that enables language service discovery, composition, and reuse and promotes sustainability, manageability, usability, and interoperability of natural language Processing (NLP) components. It is based on the {\it service-oriented architecture} (SOA), a more recent, web-oriented version of the  ``pipeline" architecture that has long been used in NLP for sequencing loosely-coupled linguistic analyses. 
The LAPPS Grid provides access to basic NLP processing tools and resources and enables pipelining such tools to create custom NLP applications, as well as
composite services such as question answering and machine translation together with language resources such as mono- and multi-lingual corpora and lexicons that support NLP. 
The transformative aspect of the LAPPS Grid is that it orchestrates  access to and deployment of language resources and processing functions available from servers around the globe and enables users to add their own language resources, services, and even service grids to satisfy their particular needs.
\end{abstract}

% Introduction

\section{Introduction}

This paper describes the basic architecture and design of the \lapps Grid API (Application Programming Interface). The \lapps Grid uses the Service Grid\footnote{\url{http://servicegrid.net}} software for user authentication, authorization, service registration and service invocation. As such the \lapps API simply defines interfaces that web services \may implement to be considered \lapps compliant.

Unlike other language service grids (Language Grid\footnote{\url{http://langrid.org}}, Panacea\footnote{\url{http://www.panacea-lr.eu}}), which define separate interfaces for each service type (translators, tokenizers, named entity recognizers, etc.) the \lapps API only defines two very simple interfaces for web services; \source and \service. The information required to invoke the underlying Language Resource (LR) is encoded as a JSON-LD\footnote{\url{http://json-ld.org}} document that is transmitted to the service.

Currently, the \lapps Grid only supports SOAP\cite{soap} web services. Future versions of the \lapps grid will also support REST based web services.


% Defined Terms


\section{Key words}
The key words "\must", "\mustnot", "\should", "\shouldnot",  "\may", and "\maynot" in this document are to be interpreted as described in
      RFC 2119\cite{rfc2119}.


% API

\section{LAPPS Application Programming Interface (API)}

The core  \lapps  API consists of one concrete class, the \data  object, and two interfaces; \source and \service.  The \data object is the main artifact exchanged by \lapps services and \source and \service are interfaces to be implemented by providers wishing to contribute services to the \lapps grid.  

% Errors

\subsection{Errors}

In the event of an error while servicing a request \lapps services \should return a \data object with the \discriminator set to \ns{error} and a UTF-8 string congaing an error message as the \payload. Services \shouldnot rely on SOAP's fault handling mechanism by throwing exceptions, either directly or indirectly. For security reasons production services \shouldnot include a stack trace in the error message.

% Data

\subsection{Data Objects}
All \lapps services communicate with each other by exchanging \data objects.  A \data object consists of a \discriminator value and a \payload.  The \data object is defined as:\footnote{All code snippets provided are in Groovy, which is (almost) a superset of the Java language.  See http://groovy.codehaus.org for an introduction to the Groovy language.}

\begin{groovy}{data}{org.lappsgrid.api.Data}
	class Data {
		String discriminator
		String payload
	}
\end{groovy}

The \discriminator value can either be a URI or a media-type conforming to the ABNF grammar in section 4.2 of RFC 6838\cite{rfc6838}. See Section~\ref{sec:media} for a discussion of recognized \lapps media types.

The \payload \should be a UTF-8 string and the \discriminator determines how the string should be interpreted.  Typically the \payload will contain a string representation of some document format, for example GATE XML or UIMA CAS.  A complete list of currently supported \discriminator types can be found in Section~\ref{sub:discriminators}. 

The following terms are used throughout the remainder of this document.

\begin{description}
\item[error object] a \data object with the \discriminator set to \ns{error}
\item[data object type] the value of the \data object's \discriminator. Sometimes referred to as the \emph{"object's type"} or \emph{"data type"} when the meaning is not ambiguous.
\end{description}

% DataSource

\subsection{DataSource}

A \source is a web service that provides data to other processing services as \data objects. A \source can return any type of \data object provided the \data object's \discriminator is set to an appropriate value.  For example, textual \source{}s will typically return \data objects with the \discriminator set to \tt{text/plain} or \tt{application/xml}. The custom attribute \emph{schema}  \may used to further describe the data format, either by referencing an actual schema document or by using a unique identifier agreed upon by service providers; for example; \tt{text/plain; schema=ptb}.

\begin{groovy}{groovy:source}{org.lappsgrid.api.DataSource}
	interface DataSource {
		Data getMetadata()
		int size()
		Data list()
		Data list(int start, int end)
		Data get(String key)
		Data query(Data input)
	}
\end{groovy}

% getMetadata
\subsubsection{Data getMetadata()}

The \tt{getMetadata()} method \should return a JSON-LD \payload with the \discriminator set to \ns{meta}. The metadata returned \should include at least the following information:
\begin{listing}
	\item name
	\item version
	\item description
\end{listing}
Section~\ref{sec:metadata} contains a complete description of the JSON-LD format for \lapps metadata.

% size
\subsubsection{int size()}
The \tt{size()} method \must return an integer value representing the number of documents managed by the \source service or a negative integer in the case of an error.  The \tt{size} is also the number of keys returned by the \tt{list()} method.

%list
\subsubsection{Data list()\\Data list(int start, int end)}

The \tt{list()} methods returns a list of the key values that can be used to fetch documents from the \source.  A \source \may return an error if the number of artifacts managed by the \source is extremely large.  In this case the \source \should return a \data object with the \discriminator set to \ns{error} and the string \emph{"Index too large."} as the \payload.

The \tt{list(int,int)} method  returns the range of key values $[start..end)$.  They keys in the returned list \must be separated  whitespace characters with the the appropriate \discriminator value from Table~\ref{table:list}.

\begin{table}[h!]
\noindent\begin{tabular}{ l p{7.5cm} }
\bold{Media type} & \bold{Description} \\
text/plain; separator=space & a list of UTF-8 strings separated by a single white space character (0x20) \\
text/plain; separator=newline & a list of UTF-8 string separated by a single newline (carridge return) character (0x0D) \\
text/plain; separator=tab & tab (0x09) separated values \\
text/plain; separator=comma  & comma (0x2C) separated values \\
\end{tabular}
\caption{Allowable media types for key lists}
\label{table:list}
\end{table}

Applications \must be able to split the returned list into an array of keys on the specified separator without doing any other text processing.

% get
\subsubsection{Data get(String key)}

The \tt{get(String)} method returns a single artifact from a \source.  The key passed to the \source \must be one of the keys returned by either of the \tt{list} methods. The \discriminator \must be set to one of the following:
\begin{listing}
\item a recognized media type listed in Section~\ref{sec:media},
\item a user defined media type defined as described in Section~\ref{sec:custom-media}, or
\item a URI referencing a user defined media type.
\end{listing}

%query
\subsubsection{Data query(Data input)}

This method is reserved for future use.  It is anticipated that \source services will be able to return documents based on a Lucene\footnote{\url{http://lucene.apache.org}} and/or Solr\footnote{\url{http://lucene.apache.org/solr}} query.  Other query languages may also be supported as well.


% WebService

\subsection{WebService}\label{sub:WebService}

A \service is any service that transforms its input, either by adding or removing annotations or metadata.

\begin{groovy}{service}{org.lappsgrid.api.WebService}
	interface WebService {
		Data getMetadata()
		Data execute(Data input)
		Data configure(Data input)
	}
\end{groovy}

% getMetadata
\subsubsection{Data getMetadata()}

The \tt{getMetadata()} method \should return a JSON-LD \payload with the \discriminator set to \ns{meta}. The metadata returned \should include at least the following information:
\begin{listing}
	\item name
	\item version
	\item description
	\item input requirements (if any),
	\item output (if any)
\end{listing}
Section~\ref{sec:metadata} contains a complete description of the JSON-LD format for \lapps metadata.

%execute
\subsubsection{Data execute(Data input)}

Executes the language resource managed by this service.  

If the input \data object's \discriminator is \ns{error} then the \service \must return the input object unchanged.  If the input \data object's \discriminator is not set to one of the specified in the service's input requirements the service \should return an error object with the payload that starts with the string "Invalid input type."

%configure
\subsubsection{configure(Data input)}

This method is reserved for future use.

%*****************
% Media types
%*****************
\section{Media Types}\label{sec:media}

Discuss media types including the following:
\begin{enum}
	\item \tt{text/plain}
	\item \tt{text/plain; separator=(space|tab|newline|comma)}
	\item \tt{text/plain; schema=<name>}
	\item \tt{application/xml}
	\item \tt{application/xml; schema=<name>}
\end{enum}

Other things to discuss:
\begin{enum}
	\item Use of custom parameters
	\begin{enum}
		\item \tt{separator} for \tt{text/plain}
		\item \tt{schema} for \tt{text/plain} and \tt{application/xml}
	\end{enum}
	\item \tt{charset} parameter ignored for all media types.
\end{enum}

% JSON-LD

\section{JSON-LD}\label{sec:json-ld}

Most of the time the content of the payload will be a JSON-LD document in the LAPPS Interchange Format (LIF).
\subsection{\lapps Interchange Format (LIF)}\label{sec:lif}

Infomation on JSON structures.

\subsection{Service Metadata}\label{sec:metadata}

Describe the JSON-LD metadata returned by the \tt{getMetadata()} method.


\section{Discriminators}\label{sec:discriminators}

\subsection{Media Types}
% WARNING: This file is machine generated.
% Any changes made to this file are likely to be lost.
\begin{longtable}{| r | l | l | p{3cm} | }
\hline \multicolumn{1}{|r|}{\textbf{ID}} & \multicolumn{1}{l|}{\textbf{Name}} & \multicolumn{1}{l|}{\textbf{URI or media-type}} & \multicolumn{1}{l|}{\textbf{Ancestors}} \\ \hline
\endhead

3 & text & text/plain &  \\ \hline
4 & xml & application/xml &  \\ \hline
5 & string-list & text/plain;separator=space &  \\ \hline
512 & one-per-line & text/plain;separator=newline & text \\ \hline
513 & tsv & text/plain;separator=tab & text \\ \hline
514 & csv & text/plain;separator=comma & text \\ \hline
1025 & gate & application/xml;profile=http://gate.ac.uk & xml, document \\ \hline
1026 & uima & application/xml;profile=http://uima.apache.org & xml, document \\ \hline
1027 & stanford & text/plain;profile=http://nlp.stanford.edu & one-per-line, document \\ \hline
1028 & opennlp & text/plain;profile=https://opennlp.apache.org & one-per-line, document \\ \hline
1029 & graf & application/xml;profile=http://graf.tc37sc4.org & xml, document \\ \hline
1030 & ptb & text/plain;profile=http://www.cis.upenn.edu/\texttildelow{}treebank & document \\ \hline
1031 & json & application/json & document \\ \hline
1032 & json-ld & application/ld+json & json \\ \hline
1034 & lapps & application/ld+json;profile=http://vocab.lappsgrid.org & json-ld \\ \hline
\end{longtable}


\subsection{License Types}\label{sub:licenses}
% WARNING: This file is machine generated.
% Any changes made to this file are likely to be lost.
\begin{longtable}{| r | l | l | p{3cm} | }
\hline \multicolumn{1}{|r|}{\textbf{ID}} & \multicolumn{1}{l|}{\textbf{Name}} & \multicolumn{1}{l|}{\textbf{URI or media-type}} & \multicolumn{1}{l|}{\textbf{Ancestors}} \\ \hline
\endhead

524287 & license & http://vocab.lappsgrid.org/ns/license &  \\ \hline
524288 & public-domain & http://vocab.lappsgrid.org/ns/license/public-domain & license \\ \hline
524289 & open-source & http://vocab.lappsgrid.org/ns/license/open-source & license \\ \hline
524290 & apache2 & http://vocab.lappsgrid.org/ns/license/apache-2.0 & open-source \\ \hline
524291 & gpl & http://vocab.lappsgrid.org/ns/license/gpl & open-source \\ \hline
524292 & lgpl & http://vocab.lappsgrid.org/ns/license/lgpl & open-source \\ \hline
524293 & bsd & http://vocab.lappsgrid.org/ns/licanse/bsd & open-source \\ \hline
524294 & eclipse & http://vocab.lappsgrid.org/ns/license/eclipse & open-source \\ \hline
524295 & no-commercial & http://vocab.lappsgrid.org/ns/license/non-commercial & license \\ \hline
524296 & restricted & http://vocab.lappsgrid.org/ns/license/restricted & license \\ \hline
524297 & gpl2 & http://vocab.lappsgrid.org/ns/license/gpl-2.0 & gpl \\ \hline
524298 & gpl3 & http://vocab.lappsgrid.org/ns/license/gpl-3.0 & gpl \\ \hline
524299 & lgpl21 & http://vocab.lappsgrid.org/ns/license/lgpl-2.1 & lgpl \\ \hline
524300 & lgpl3 & http://vocab.lappsgrid.org/ns/license/lgpl-3.0 & lgpl \\ \hline
524301 & bsd2 & http://vocab.lappsgrid.org/ns/license/bsd-2-clause & bsd \\ \hline
524302 & bsd3 & http://vocab.lappsgrid.org/ns/license/bsd-3-clause & bsd \\ \hline
524303 & mit & http://vocab.lappsgrid.org/ns/license/mit & open-source \\ \hline
\end{longtable}


\subsection{All Discriminators}\label{sub:discriminators}
% WARNING: This file is machine generated.
% Any changes made to this file are likely to be lost.
\begin{longtable}{| r | l | l | p{3cm} | }
\hline \multicolumn{1}{|r|}{\textbf{ID}} & \multicolumn{1}{l|}{\textbf{Name}} & \multicolumn{1}{l|}{\textbf{URI or media-type}} & \multicolumn{1}{l|}{\textbf{Ancestors}} \\ \hline
\endhead

0 & error & http://ns.lappsgrid.org/1.0/error &  \\ \hline
1 & ok & http://ns.lappsgrid.org/1.0/ok &  \\ \hline
2 & meta & http://ns.lappsgrid.org/1.0/meta &  \\ \hline
3 & text & text/plain &  \\ \hline
4 & xml & application/xml &  \\ \hline
5 & string-list & text/plain;separator=space &  \\ \hline
512 & one-per-line & text/plain;separator=newline & text \\ \hline
513 & tsv & text/plain;separator=tab & text \\ \hline
514 & csv & text/plain;separator=comma & text \\ \hline
1024 & document & http://vocab.lappsgrid.org/Document &  \\ \hline
1025 & gate & application/xml;profile=http://gate.ac.uk & xml, document \\ \hline
1026 & uima & application/xml;profile=http://uima.apache.org & xml, document \\ \hline
1027 & stanford & text/plain;profile=http://nlp.stanford.edu & one-per-line, document \\ \hline
1028 & opennlp & text/plain;profile=https://opennlp.apache.org & one-per-line, document \\ \hline
1029 & graf & application/xml;profile=http://graf.tc37sc4.org & xml, document \\ \hline
1030 & ptb & text/plain;profile=http://www.cis.upenn.edu/\texttildelow{}treebank & document \\ \hline
1031 & json & application/json & document \\ \hline
1032 & json-ld & application/ld+json & json \\ \hline
1034 & lapps & application/ld+json;profile=http://vocab.lappsgrid.org & json-ld \\ \hline
2048 & annotation & http://vocab.lappsgrid.org/Annotation &  \\ \hline
2049 & chunk & http://vocab.lappsgrid.org/Chunk & annotation \\ \hline
2050 & paragraph & http://vocab.lappsgrid.org/Paragraph & chunk \\ \hline
2051 & sentence & http://vocab.lappsgrid.org/Sentence & chunk \\ \hline
2052 & token & http://vocab.lappsgrid.org/Token & chunk \\ \hline
2053 & pos & http://vocab.lappsgrid.org/Token\#pos & annotation \\ \hline
2054 & coref & http://vocab.lappsgrid.org/NamedEntity\#matches & annotation \\ \hline
2055 & ne & http://vocab.lappsgrid.org/NamedEntity & annotation \\ \hline
2056 & person & http://vocab.lappsgrid.org/Person & ne \\ \hline
2057 & location & http://vocab.lappsgrid.org/Location & ne \\ \hline
2058 & date & http://vocab.lappsgrid.org/Date & ne \\ \hline
2059 & organization & http://vocab.lappsgrid.org/Organization & ne \\ \hline
2060 & nchunk & http://vocab.lappsgrid.org/NounChunk & chunk \\ \hline
2061 & vchunk & http://vocab.lappsgrid.org/VerbChunk & chunk \\ \hline
2062 & lemma & http://vocab.lappsgrid.org/Token\#lemma & annotation \\ \hline
2063 & lookup & http://vocab.lappsgrid.org/Lookup & annotation \\ \hline
2064 & matches & http://vocab.lappsgrid.org/NamedEntity\#matches & annotation \\ \hline
3072 & query & http://ns.lappsgrid.org/1.0/query &  \\ \hline
3073 & get & http://ns.lappsgrid.org/1.0/query\#get &  \\ \hline
3074 & index & http://ns.lappsgrid.org/1.0/query\#index &  \\ \hline
3075 & list & http://ns.lappsgrid.org/1.0/query\#list &  \\ \hline
3076 & lucene & http://ns.lappsgrid.org/1.0/query\#lucene & query \\ \hline
3077 & sql & http://ns.lappsgrid.org/1.0/query\#sql & query \\ \hline
3078 & sparql & http://ns.lappsgrid.org/1.0/query\#sparql & query \\ \hline
3079 & regex & http://ns.lappsgrid.org/1.0/query\#regex & query \\ \hline
3080 & composite & http://ns.lappsgrid.org/1.0/query\#composite &  \\ \hline
524287 & license & http://ns.lappsgrid.org/1.0/license &  \\ \hline
524288 & public-domain & http://ns.lappsgrid.org/1.0/license\#public-domain & license \\ \hline
524289 & open-source & http://ns.lappsgrid.org/1.0/license\#open-source & license \\ \hline
524290 & apache2 & http://ns.lappsgrid.org/1.0/license\#apache-2.0 & open-source \\ \hline
524291 & gpl & http://ns.lappsgrid.org/1.0/license\#gpl & open-source \\ \hline
524292 & lgpl & http://ns.lappsgrid.org/1.0/license\#lgpl & open-source \\ \hline
524293 & bsd & http://ns.lappsgrid.org/1.0/license\#bsd & open-source \\ \hline
524294 & eclipse & http://ns.lappsgrid.org/1.0/license\#eclipse & open-source \\ \hline
524295 & no-commercial & http://ns.lappsgrid.org/1.0/license\#non-commercial & license \\ \hline
524296 & restricted & http://ns.lappsgrid.org/1.0/license\#restricted & license \\ \hline
524297 & gpl2 & http://ns.lappsgrid.org/1.0/license\#gpl-2.0 & gpl \\ \hline
524298 & gpl3 & http://ns.lappsgrid.org/1.0/license\#gpl-3.0 & gpl \\ \hline
524299 & lgpl21 & http://ns.lappsgrid.org/1.0/license\#lgpl-2.1 & lgpl \\ \hline
524300 & lgpl3 & http://ns.lappsgrid.org/1.0/license\#lgpl-3.0 & lgpl \\ \hline
524301 & bsd2 & http://ns.lappsgrid.org/1.0/license\#bsd-2-clause & bsd \\ \hline
524302 & bsd3 & http://ns.lappsgrid.org/1.0/license\#bsd-3-clause & bsd \\ \hline
524303 & mit & http://ns.lappsgrid.org/1.0/license\#mit & open-source \\ \hline
525311 & usage & http://ns.lappsgrid.org/1.0/allow &  \\ \hline
525312 & research & http://ns.lappsgrid.org/1.0/allow\#research & usage \\ \hline
525313 & commercial & http://ns.lappsgrid.org/1.0/allow\#commercial & usage \\ \hline
525314 & education & http://ns.lappsgrid.org/1.0/allow\#education & usage \\ \hline
525315 & non-profit & http://ns.lappsgrid.org/1.0/allow\#non-profit & usage \\ \hline
525316 & personal & http://ns.lappsgrid.org/1.0/allow\#personal & usage \\ \hline
525317 & any & http://ns.lappsgrid.org/1.0/allow\#any & non-profit, research, personal, education, commercial \\ \hline
\end{longtable}



\bibliographystyle{plain}
\bibliography{design}

\end{document}