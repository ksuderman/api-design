
%*****************
% Media types
%*****************
\section{Media Types}\label{sec:media}

\lapps services describe media types using the ABNF grammar  defined in Sectio~4.2 of RFC~6838\cite{rfc6838}. However, not all media types defined using this grammar are valid \lapps media types.  In particular, the only media type/subtype combinations currently recognized by \lapps services are:
\begin{listing}
\item \tt{text/plain},
\item \tt{application/xml},
\item \tt{application/json}, and
\item \tt{application/ld+json}
\end{listing}

Services \may recognize other media types; these are simply the ones recognized by current \lapps services.

The \lapps API also defines two custom parameters that can be included with the media type.

\begin{description}
\item[separator] The separator character used between entries in a list of data. Possible values are \it{space} (0x20), \it{tab} (0x09), \it{newline} (0x0D), and \it{comma} (0x2C). This parameter only applies to the \tt{text/plain} media type.
\item[profile] Describes the format of structured data.  The parameter value \may point to an actual schema document, or may simply use a recognized term agreed upon by service providers.  The currently recognized terms are:
\begin{enum}
	\item \tt{text/plain; profile=http://vocab.lappsgrid.org/ns/media\#ptb} for the lisp-like bracketing used by the Penn TreeBank.
	\item \tt{application/xml; profile=http://vocab.lappsgrid.org/ns/media\#gate} for GATE XML documents.
	\item \tt{application/xml; profile=http://vocab.lappsgrid.org/ns/media\#uima} for UIMA CAS documents.
	\item \tt{application/xml; profile=http://vocab.lappsgrid.org/ns/media\#graf} for GrAF standoff annotation files.
	\item \tt{application/ld+json; profile=http://vocab.lappsgrid.org/ns/media\#lapps} for JSON-LD documents conforming to the \lapps Interchange Format.
	\item \tt{application/ld+json; profile=http://vocab.lappsgrid.org/ns/media\#lapps-meta} for JSON-LD documents contains \lapps service metadata.
\end{enum}
\end{description}

\todo{Mention that that charset parameter \must be ignored."}

Discuss media types including the following:
\begin{enum}
	\item \tt{text/plain}
	\item \tt{text/plain; separator=(space|tab|newline|comma)}
	\item \tt{text/plain; profile=<name>}
	\item \tt{application/xml}
	\item \tt{application/xml; profile=<name>}
\end{enum}

Other things to discuss:
\begin{enum}
	\item Use of custom parameters
	\begin{enum}
		\item \tt{separator} for \tt{text/plain}
		\item \tt{profile} for \tt{text/plain} and \tt{application/xml}
	\end{enum}
	\item \tt{charset} parameter ignored for all media types.
\end{enum}

